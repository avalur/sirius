\course{Введение в рекуррентные нейронные сети}{А.~Ю.~Авдюшенко}

Возможно вы игрались с сервисом \link{https://yandex.ru/lab/yalm}{<<Балабоба>>}, который позволяет генерировать классные тексты по заданным нескольким первым словам.

Под капотом у него разработанная командой Яндекса языковая модель YaLM (Yet another Language Model), ядром которой является рекуррентная нейронная сеть на архитектуре Transformer.

В предлагаемом мини-курсе вы научитесь:
\begin{itemize}
\item основам языка Python и Jupyter-ноутбуков
\item писать там свою простейшую (vanilla) рекуррентную нейронную сеть
\item устройству LSTM-модулей, поймете зачем они нужны
\item генерировать тексты рассказов или песен вашего любимого автора
\end{itemize}

\paragraph{Рекомендации для подготовки}
\begin{itemize}
    \item Познакомиться с языком программирования Python. Для этого можно пройти один курс из списка ниже или аналогичный на ваш вкус.
    \begin{itemize}
        \item \link{https://stepik.org/course/67}{Программирование на Python}
        \item Ещё вариант \link{https://stepik.org/course/238}{<<Introduction to Python>>} с использованием IDE PyCharm
    \end{itemize}
    \item Познакомиться с библиотекой numpy, например, \link{http://cs231n.github.io/python-numpy-tutorial/}{тут}
\end{itemize}

\lecturer{avdyushenko.jpg}{Доцент и руководитель программы <<\href{https://maad.compscicenter.ru/}{Науки о данных}>> факультета МКН \text{СПбГУ}, куратор CS центра, кандидат физико-математических наук, выпускник ШАД. В прошлом — аналитик Яндекс.Справочника. Увлечён машинным обучением и обучением людей. На МКН \text{СПбГУ} ведёт курсы <<Машинное обучение>> и <<Язык программирования Python>>.}{Александр Юрьевич Авдюшенко}{\text{СПбГУ}, ШАД Яндекса}
